\begin{frame}
~\\
{\blank
\underline{RECAP:}\\
\underline{Aim:} Given $A\in\mathbb{R}^{m\times n}~(m\neq n),~b\in\mathbb{R}^m$, find $x\in\mathbb{R}^n$, such that $$Ax=b.$$
\underline{Our current approach:} Direct method (= finitely many steps)\\
\underline{Idea:} Use the fact $ax = b~~\Leftrightarrow~~\underbrace{TA}_{\tilde{A}}x=\underbrace{Tb}_{\tilde{b}}~~\forall~T\in GL_m(\mathbb{R})$
($''\Rightarrow''$: $T$ linear and $''\Leftarrow''$: $T$ injective)\\
~\\
The goal is to apply multiple $T$'s so that ultimately $\tilde{A}$ is ''triangular'' ($\rightarrow$ easy to solve).\\
To reach this goal we apply the following two elementary operations:
\begin{itemize}\blank
	\item [1)]
	Add a nonzero multiple of one row to multiple other rows:
	$$
	L_{\textcolor{cyan}{j}} = \begin{pmatrix}
	1&0&\cdots&~&0\\
	0&\ddots&0&\cdots&\vdots\\
	\vdots&0&\textcolor{cyan}{1}&0&~\\
	~&\vdots&\textcolor{cyan}{|}&\ddots&0\\
	0&0&\textcolor{cyan}{|}&0&1
	\end{pmatrix}~~~\rightarrow~\text{Frobenius matrices}
	$$
	\item [2)]
	Row swap:
	$$
	\rightarrow~~P~\text{permutation matrix}
	$$
\end{itemize}
}
\end{frame}
\begin{frame}
	~\\
	{\blank
		\underline{RECAP:}\\
		\underline{Solve:} $Ax=b,~A\in\mathbb{R}^{m\times n},~b\in\mathbb{R}^m$\\
		\underline{Apply:} $L_j,~P_{jk_j}~~\rightsquigarrow~~Ux=z$,
		In fact: $(m=n)$
		\begin{align*}
		&\underbrace{
			L_{(n-1)}P_{(n-1)k_{(n-1)}}\dots L_2P_{2k_2}L_1P_{1k_1}}_{=\underbrace{
				(\hat{L}_{(n-1)}\cdot\dots\cdot\hat{L}_1)}_{=:\tilde{L}}\underbrace{
				(P_{(n-1)k_{(n-1)}}\cdot\dots\cdot P_{1k_1})}_{=:P}}\cdot A=U~~(\leftarrow~\text{REF})\\
		\Leftrightarrow~~&\tilde{L}PA=U
		\stackrel{L:=\tilde{L}^{-1}}{\Leftrightarrow}~~PA = LU~~(LU-\text{decomposition of}~A~\text{with row pivoting})
		\end{align*}
		\begin{itemize}\blank
			\item [1)]
			\underline{Frobenius matrix:}
			$L_j = I+l_je_j^T\in\mathbb{R}^{m\times m},~l_j:=~(0,\dots,0,l_{j+1,j},\dots,l_{m,j})^T\in\mathbb{R}^m$
			\begin{itemize}
				\item [i)]
				$L_j^{-1}=I-l_je_j^T$
				\item [ii)]
				$L_i\cdot L_j = I+l_je_j^T+l_ie_i^T$
			\end{itemize}
			\item [2)]
			\underline{Permutation matrix:}\\
			$P\in\mathbb{R}^{m\times m}$ is called permutation matrix if it has exactly one entry ''1'' in each row and column
			$$
			P=(p_{ij})_{ij}=\begin{pmatrix}
			0&\cdots&1&0\\
			0&1&\cdots&0\\
			~&~&\vdots&~\\
			0&\cdots&0&1
			\end{pmatrix}~
			{\cyan \begin{bmatrix}
				p_1\\p_2\\\vdots\\p_m
				\end{bmatrix},~p_i:=~\text{gives us the column of the 1 in the i-th row}}
			$$
			$p_{ij}:=\lbrace\begin{matrix}
			1:~{\cyan p_i=j}\\
			0:~\text{else}
			\end{matrix}$, Bsp.: $P = \begin{pmatrix}
			0&1&0\\1&0&0\\0&0&1
			\end{pmatrix}{\cyan \begin{bmatrix}
				2\\1\\3
				\end{bmatrix}}$
		\end{itemize}
	}
\end{frame}